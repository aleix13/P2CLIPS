\section{Identificació del problema}
Hem d'iniciar una primera fase on analitzarem el problema que tenim entre mans. Per a fer-ho analitzarem la viabilitat de construir un SBC, les fonts de coneixement i els objectius i resultats del sistema, però primer descriurem el problema.

\subsection{Descripció del Problema}
Una cadena de gimnasos, \textit{I'm no couch potatoe}, vol desenvolupar un sistema basat en el coneixement que pugui recomanar programes d'exercicis als seus clients. Per a fer-ho disposem d'informació variada del client en qüestió:

\begin{itemize}
    \item Dades bàsiques del client, com vindrien a ser el pes, l'altura (amb aquests dos podem calcular l'índex de massa corporal), l'edat o la pressió sanguínia. Amb això podem començar a elaborar un perfil del nostre client.
    \item Hàbits del client que puguin ser rellevants pel que fa l'exercici físic. Aquestes poden ser totes les activitats físiques que realitza en l'àmbit domèstic, com podrien ser planxar, fregar... També ens poden interessar les activitats de tipus laboral, si aquestes involucren esforç físic, com podria ser aixecament d'objectes pesats, desplaçaments a peu o amb bicicleta o bé accions repetitives, esforços musculars, entre d'altres. Totes aquestes ens poden ajudar a veure amb quin nivell hem de començar.
    \item Dades de salut del client. Aquestes podrien ser algun problema muscular o esquelètics, com mal l'esquena, problemes a les cervicals, problemes articulatoris, problemes respiratoris, cardíacs... També en aquest sentit ens interessa la dieta de l'usuari, saber coses com si abusa de la sal, o si té una dieta equilibrada on hi incorpora fruita i verdura, si té mals hàbits com picar entre hores... Tot això pot determinar en la forma inicial d'aquest.
    \item Dades del programa d'entrenament. Hem de saber de quant temps disposa el client per entrenar, així com quin o quins són els seus objectius, per exemple aprimar-se, guanyar massa muscular, mantenir-se, tenir millor condició física o guanyar elasticitat. 
\end{itemize}

No només tenim informació del client sinó que també tenim informació sobre els diferents exercicis que podem assignar als clients. Per a cada un d'ells tenim:

\begin{itemize}
    \item Objectiu pel qual estan pensats, això no exclou els altres però si que indica idoneïtat per alguns objectius en concret.
    \item Grups musculars que estimulen. Cada exercici pot tenir varis grups musculars on tenen incidència.
    \item Contraindicacions. Certs exercicis poden no estar recomanats en certs casos. A mode d'exemple, tenir problemes cardíacs pot ser una contraindicació en certs casos, també problemes de pressió alta, dolors a l'esquena... També en molts casos pot ser determinant l'edat, hi pot haver exercicis poc adequats per menors d'edat o per persones grans.
    
\end{itemize}

El sistema ha de generar un programa d'entrenament pel client. Aquest ha de ser complet, és a dir, ha de generar exercicis que satisfacin l'objectiu del client i que treballin en completesa tot el cos. També han de ser variats, s'ha d'evitar la monotonia ja que s'ha d'aconseguir una experiència divertida per l'usuari, que es diverteixi mentre faci esport. Això implica molts i diferents exercicis per dia i també per setmana.

\subsection{Viabilitat d'abordar el problema amb un SBC}

\subsection{Fonts de coneixement}
\subsection{Objectius del problema i resultats del sistema}